\subsubsection{Dirichlet Convolution}
    $\varepsilon = \mu * 1$\\
    $\varphi = \mu * \text{Id}$
\subsubsection{Burnside's Lemma}
Let $X$ be a set and $G$ be a group that acts on $X$.
For $g \in G$, denote by $X^g$ the elements fixed by $g$:
\[
X^g = \{ x \in X \mid gx \in X \}
\]
Then
\[
|X/G| = \frac{1}{|G|} \sum_{g \in G} |X^g|.
\]
\subsubsection{Pick Theorem}
    $\text{Area} = \text{inner lattice point} + \frac{\text{lattice point on border}}{2} - 1$
\subsubsection{Fermat's Little Theorem}
    $(a+b)^p \equiv a+b \equiv a^p+b^p \pmod p$
\subsubsection{Wilson's Theorem}
    $(p-1)! \equiv -1 \pmod p$
\subsubsection{Legendre Theorem}
    $v_p(n):=\text{power of $p$ in $n$}$\\
    $(n)_p:=\frac{n}{p^{(v_p(n))}}\\$
    $s_p(n):=$sum of all digits of $n$ in base $p$\\
    $v_p(n!)=\sum_{i=1}^\infty \lfloor \frac n {p^i} \rfloor=\frac{n-s_p(n)}{p-1}$
\subsubsection{Kummer Theorem}
    $v_p(\binom{n}{m})=\frac{s_p(n)+s_p(m-n)-s_p(m)}{p-1}$
\subsubsection{ext-Kummer Theorem}
    $v_p(\binom{n}{{m_1,m_2,...m_k}})=\frac{\sum_{i=1}^k s_p(m_i)-s_p(n)}{p-1}$
\subsubsection{Factorial with mod}
    $(n!)_p \equiv -1^{\lfloor \frac n p\rfloor}((\lfloor \frac n p\rfloor)!)_p((n\%p)!) \pmod p$
    $O(p+\log_p(n))$ with factorial table.
\subsubsection{Properties of nCr with mod}
    If any i in base p satisfies $n_i<m_i$, then $\binom{n_i}{m_i}\%p=0$.
    Therefore $\binom{n}{m}=\prod_{i=0}^{\max(\log_p(a),\log_p(b))}\binom{n_i}{m_i}\%p$ so $\binom{n}{m}\%p=0$.
    If $p=2$, then $\binom{n}{m}$ is odd $\Leftrightarrow$ any bit in $n<m$.
    Lucas' theorem can be derived from this generating function method without relying on Fermat's Little Theorem.
    It is also true for polynomials.
\subsubsection{ext-Lucas' Theorem}
    For any $k\in$ positive number, calculate $\binom{n}{m}\%k$ can decompose k by Fundamental Theorem of Arithmetic.
    And then use crt.
\subsubsection{Catalan Number}
    $C_0=C_1=1, \text{if }n>1\text{then }C_n=\sum_{k=0}^{n-1}C_kC_{n-1-k}=\frac{\binom{2n}{n}}{n+1}$
    Also the number of legal placements of n pairs of brackets is $C_n$.
    If there are any k kinds of brackets available, then $k^nC_n$.
\subsubsection{modinv table}
    $p=i*(p/i)+p\%i,-p\%i=i*(p/i),inv(i)=-(p/i)*inv(p\%i)$
\subsubsection{LTE}
\[
\begin{array}{c}
    p\text{ is odd prime},n\in\mathbb{N},x\in\mathbb{Z},y\in\mathbb{Z}\\
    p|(x-y),p \nmid x,p \nmid y\\
    v_p(x^n-y^n)=v_p(x-y)+v_p(n)\\
    \begin{matrix}
    v_2(x^n-y^n)=v_2(x-y)+v_2(n)&\text{if $n$ is odd}\\
    v_2(x^n-y^n)=v_2(x-y)+v_2(x+y)+v_2(n)-1&\text{else}
    \end{matrix} \\
    4|(x-y)\Rightarrow v_2(x+y)=1,v_2(x^n-y^n)=v_2(x-y)+v_2(n)
\end{array}
\]